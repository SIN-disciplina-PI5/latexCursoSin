\documentclass[a4paper,12pt]{report}
\usepackage[utf8]{inputenc}
\usepackage{graphicx}
\usepackage[brazil]{babel}
\usepackage{hyperref}
\usepackage{newtxtext} 
\usepackage{newtxmath} 
\usepackage[letterpaper,top=2cm,bottom=2cm,left=3cm,right=3cm,marginparwidth=1.75cm]{geometry}
\usepackage{amsmath}

% Configurações da capa
\title{Relatório sobre a Página do Curso}
\author{
    Alex Lucas, Arthur Eulálio, Gabriel Albuquerque, José Gabriel \and
    Nicolly Mendonça, Paulo Barreto, Sâmya Maria e Tatiana Helena.
}
\date{\today}

\begin{document}

% Capa
\maketitle

% Resumo
\begin{abstract}
Este relatório apresenta uma análise abrangente do desenvolvimento e implementação do site do curso de Sistemas para Internet. A análise detalha a arquitetura do sistema, suas principais funcionalidades, componentes, e a interação entre o Front-end e o Back-end. O objetivo principal é fornecer uma visão clara e aprofundada das características mais relevantes do site, incluindo a centralização de informações, a divulgação de artigos científicos, e a facilitação da comunicação entre alunos e professores.  
 
Além disso, o relatório avalia o impacto da plataforma na experiência do usuário, destacando melhorias na acessibilidade da informação, no engajamento dos alunos, e na eficiência da comunicação institucional. Através desta análise, busca-se demonstrar como a implementação do site contribuiu para o aprimoramento das atividades acadêmicas e administrativas do curso. 
\end{abstract}

% Sumário
\tableofcontents
\newpage

% Introdução
\chapter{Introdução}
O presente trabalho propõe o desenvolvimento de um site para o curso de Tecnologia, visando aprimorar a disseminação de informações e facilitar o acesso a recursos importantes para os alunos e professores. A falta de uma plataforma centralizada tem dificultado a comunicação eficiente, tornando-se necessária a criação de um sistema que agregue todas as informações relevantes de maneira organizada e acessível.

\section{Contextualização e Motivação}
Atualmente, as informações relacionadas ao curso de Tecnologia estão dispersas em diferentes canais, dificultando o acesso e a comunicação eficiente entre os membros da comunidade acadêmica. Essa dispersão de informações pode levar à desinformação e à perda de oportunidades importantes para os alunos. A motivação para este projeto surge da necessidade de criar uma plataforma única e acessível, onde todas as informações pertinentes ao curso possam ser facilmente encontradas.

\section{Problemática}
A principal problemática enfrentada pela comunidade acadêmica do curso de Tecnologia é a dificuldade de acesso a informações centralizadas e atualizadas. A falta de um ponto de referência único resulta em comunicação ineficaz, perda de informações importantes e dificuldade de acompanhamento das atividades do curso. Esse problema afeta diretamente a qualidade da experiência acadêmica dos alunos e a eficiência na administração do curso.

\section{Objetivos}
O desenvolvimento do site tem como objetivo geral centralizar as informações e melhorar a comunicação entre os membros da comunidade acadêmica. Para alcançar esse objetivo, foram definidos os seguintes objetivos específicos:

\subsection{Objetivo Geral}
Desenvolver um site informativo para o curso de Tecnologia, que centralize todas as informações importantes, projetos de extensão e avisos relevantes. 

\subsubsection{Objetivos Específicos}
\begin{itemize}
    \item Criar uma plataforma de fácil navegação e acesso intuitivo.
    \item Disponibilizar informações detalhadas sobre o curso, como grade curricular, corpo docente e projetos de extensão.
    \item Implementar uma seção de avisos e notícias atualizada regularmente.
    \item Incluir funcionalidades que facilitem a comunicação entre alunos e professores.
\end{itemize}

% Funcionalidades da Página do Curso
\chapter{Metodologia}
\section{Arquitetura de Software} 
A arquitetura utilizada em Front-end e Back-end são distintas, com comunicação via API. O padrão adotado é o MVC (Model-View-Controller). O front-end (View) interage com o back-end (Controller) que manipula os dados (Model). 

\section{Detalhes Técnicos}
\subsection{Front-End} 
O Front-end é implementado em React. 
Sobre os componentes do sistema: 
\begin{itemize} 
    \item \textbf{Página Inicial:} Um componente que exibe artigos e a home do site. 
    \item \textbf{Login:} Um componente para realizar login com campos de entrada para Email e senha. 
    \item \textbf{Corpo Docente:} Componentes que mostram os docentes e informações sobre o curso. 
\end{itemize}

\subsection{Roteamento} 
As rotas no Front-end são manipuladas pelo react-router-dom, uma biblioteca para roteamento no React. 

\subsection{Tratamento de Erros} 
Se o login falhar, a resposta da API é tratada, e a mensagem de erro retornada pelo servidor é exibida ao usuário. Se houver um erro de conexão ao servidor, a mensagem "Erro ao conectar ao servidor" é exibida. 

\subsection{Notas e Observações} 
\begin{itemize} 
    \item O código é bem estruturado e segue boas práticas de React, como uso de Hooks (useState, useNavigate) e tratamento de erros. 
\end{itemize} 

\subsection{Back-End} 
Nosso Back-end é feito em .NET Core 8.0, Entity Framework Core e Padrão de Arquitetura limpa CQRS (Segregação de Responsabilidade de Comando e Consulta).
  
Através da utilização de rotas expressas, as solicitações HTTP que correspondem a determinadas rotas passam por uma camada de verificação chamada Middleware de CORS antes de alcançarem a camada de segurança. 
A camada de segurança é composta por: 
\begin{itemize} 
  \item Middleware de autenticação JWT: responsável por verificar a validade da inscrição e autenticidade do token. 
  \item Middleware de autorização: verifica as permissões do usuário por meio de informações registradas no banco de dados. 
\end{itemize} 
Caso algum erro seja detectado por esses middlewares, uma mensagem adequada é enviada como resposta HTTP. 
Os controladores, por sua vez, estabelecem a interação com o banco de dados SQLServer utilizando o Entity Framework Core e enviam respostas HTTP ao cliente, como tokens de autenticação, informações do usuário e dados específicos com base em funções de acesso. 

\section{Funcionalidades e Casos de Uso}
 \subsection{Descrição Geral}
    \label{sec:descricao}
O projeto do site do curso de graduação em Sistemas para Internet foi motivado pela necessidade de centralizar e organizar informações importantes, divulgar avisos, fornecer informações sobre o curso e o corpo docente, e facilitar a divulgação de artigos pelos professores. A plataforma visa melhorar a comunicação e o acesso a informações para alunos e professores, além de promover a visibilidade do curso.
Nossa plataforma online contará com cinco seções principais, que terão suas funcionalidades descritas a seguir.
 
\begin{itemize} 
    \item Sobre o Curso: Esta seção fornecerá uma visão geral do curso de Sistemas para Internet, será um recurso importante para futuros alunos e aqueles que desejam mais informações sobre o curso.
 
    \item Corpo Docente: Aqui serão apresentadas informações sobre os professores do curso, incluindo suas qualificações, áreas de especialização, e artigos. Esta seção permitirá aos alunos conhecerem melhor seus professores e suas áreas de atuação.
 
    \item Avisos e Comunicados: Esta seção será utilizada para divulgar avisos importantes, como mudanças no cronograma, eventos do curso, prazos de inscrição, e outras informações relevantes. Será uma ferramenta essencial para manter os alunos atualizados sobre as atividades do curso.
 
    \item Divulgação de Artigos: Exclusiva para professores, esta seção permitirá a divulgação de artigos científicos e publicações realizadas pelo corpo docente. Professores poderão compartilhar suas pesquisas e contribuições acadêmicas, promovendo a troca de conhecimento e estimulando o interesse dos alunos pela pesquisa científica.
 
    \item Conta e Login: Os usuários poderão criar uma conta no site para ter acesso personalizado. Alunos e professores terão áreas dedicadas, onde poderão acessar funcionalidades específicas. Algumas funcionalidades, como a publicação de artigos, serão restritas aos professores, garantindo a integridade e qualidade das informações divulgadas.
\end{itemize}

\subsection{Possíveis Casos de Uso}
 
\subsubsection{Caso de Uso 1}
 
Objetivo do usuário: Realizar Cadastro e acessar área Aluno do site.
 
Como apresentado na Figura, após acessar o site, o usuário poderá optar por realizar o cadastro. Após o preenchimento dos dados necessários e a confirmação do cadastro, o usuário poderá fazer login na página inicial. Uma vez autenticado, o usuário terá acesso à área Aluno do site, onde poderá visualizar informações exclusivas, como artigos publicados pelos professores, materiais de estudo e avisos importantes.
 
\subsubsection{Caso de Uso 2}
 
Objetivo do usuário: Acessar e visualizar informações sobre extensões do curso.
 
Como apresentado na Figura, ao navegar pelo site, o usuário poderá acessar a seção de "Extensões do Curso". Nessa seção, estarão listadas todas as atividades de extensão oferecidas pelo curso, como workshops, palestras, eventos e projetos de pesquisa em andamento. Facilitando assim sua participação em eventos e iniciativas extracurriculares relacionadas ao curso.
 
\subsubsection{Caso de Uso 3}
 
Objetivo do usuário: Visualizar perfil do corpo docente e entrar em contato.
 
Como apresentado na Figura, ao acessar a seção "Corpo Docente" do site, o usuário poderá visualizar informações sobre os professores do curso, como nome, disciplinas lecionadas, áreas de pesquisa, e-mail de contato, entre outros.

\section{Diagrama de Classes}
 
\subsection{Descrição do Diagrama}
 
O site do curso de Sistema para Internet visa fornecer uma plataforma interativa e informativa para alunos, professores e interessados no curso. O diagrama de classes apresentado abaixo ilustra a estrutura dos principais componentes do sistema e suas interações.
 
\subsection{Curso de Sistema para Internet}
 
O curso de Sistema para Internet é o elemento central do sistema, representado por uma classe que armazena informações essenciais, como nome do curso, descrição, coordenador, corpo docente, e outros detalhes relevantes.
 
\subsection{Usuários}
 
Os usuários do sistema podem ser divididos em duas categorias principais: alunos e professores. Cada usuário possui informações específicas, como nome, email e senha. A classe de usuário também inclui métodos para autenticação e controle de acesso ao sistema.
 
\subsection{Avisos e Informações}
 
Esta classe representa os avisos e informações que são divulgados no site. Pode incluir anúncios importantes, notificações de eventos, prazos de inscrição, entre outros. Cada aviso possui um título, conteúdo, data de publicação e autor.
 
\subsection{Extensões do Curso}
 
As extensões do curso referem-se a atividades extracurriculares oferecidas aos alunos, como workshops, palestras, eventos e projetos de pesquisa. Cada extensão possui informações detalhadas, como título, descrição, datas, horários, localização e responsáveis pela organização.
 
\subsection{Artigos}
 
Os artigos são materiais acadêmicos ou informativos escritos pelos professores do curso. Cada artigo pode conter um título, resumo, conteúdo completo, data de publicação e autor. Os alunos podem acessar e visualizar esses artigos para obter conhecimento adicional sobre tópicos relevantes.
 
\subsection{Relacionamentos}
 
O diagrama de classes também inclui relacionamentos entre as diferentes entidades do sistema. Por exemplo, um usuário pode estar associado a vários avisos e extensões do curso, enquanto uma extensão do curso pode ter vários alunos inscritos.
 
\subsection{Objetivos do Diagrama}
 
O diagrama de classes tem como objetivo fornecer uma visão geral da estrutura do sistema e das relações entre seus componentes. Ele serve como um guia para o desenvolvimento e implementação do site do curso de Sistema para Internet, garantindo uma organização eficiente e uma experiência de usuário satisfatória.

\subsection{Problemas Identificados no Contexto do Curso}
 
Em resumo, os principais problemas identificados no contexto do curso de Sistema para Internet consistem em:
\begin{itemize}
    \item Falta de acesso à informação adequada sobre o curso, suas disciplinas, atividades extracurriculares e eventos relacionados.
    \item Ausência de uma solução efetiva e durável para estimular a participação dos alunos em atividades acadêmicas e extracurriculares, bem como para promover a interação entre alunos e professores.
\end{itemize}
 
\section{Relações entre as Classes} 
  
\subsection{Notas e Observações} 
  
\begin{itemize} 
  \item Conscientizar os alunos sobre a importância do curso de Sistema para Internet e os benefícios de uma formação nessa área. 
  \item Disponibilizar informações atualizadas sobre o curso, como grade curricular, disciplinas oferecidas, corpo docente, entre outros, para que os interessados possam tomar decisões informadas. 
  \item Implementar elementos interativos no site para promover maior engajamento dos alunos. 
\end{itemize} 

\subsection{Back-End} 
Nosso Back-end é feito em .NET Core 8.0, Entity Framework Core e Padrão de Arquitetura limpa CQRS (Segregação de Responsabilidade de Comando e Consulta).
  
Através da utilização de rotas expressas, as solicitações HTTP que correspondem a determinadas rotas passam por uma camada de verificação chamada Middleware de CORS antes de alcançarem a camada de segurança. 
A camada de segurança é composta por: 
\begin{itemize} 
  \item Middleware de autenticação JWT: responsável por verificar a validade da inscrição e autenticidade do token. 
  \item Middleware de autorização: verifica as permissões do usuário por meio de informações registradas no banco de dados. 
\end{itemize} 
Caso algum erro seja detectado por esses middlewares, uma mensagem adequada é enviada como resposta HTTP. 
Os controladores, por sua vez, estabelecem a interação com o banco de dados SQLServer utilizando o Entity Framework Core e enviam respostas HTTP ao cliente, como tokens de autenticação, informações do usuário e dados específicos com base em funções de acesso. 

\chapter{Resultados}
\section{Resultados Alcançados}
Os resultados que serão alcançados com a implementação do site do curso de Sistemas para Internet são extremamente positivos. A centralização das informações, a facilitação da comunicação e o aumento do engajamento dos alunos são apenas algumas das melhorias observadas. Esses resultados demonstram que a plataforma irá atingir seus objetivos propostos, beneficiando toda a comunidade acadêmica do curso.

\section{Estudos de Caso - Simulação}
Dois estudos de caso foram realizados para ilustrar os impactos do site na vida acadêmica dos alunos e professores:

\subsection{Estudo de Caso 1: Aumento na Participação em Workshops e Atividades Extracurriculares}
Após a implementação do site, a divulgação de workshops e eventos de extensão tornou-se mais eficiente. Como resultado, foi observado um aumento na participação dos alunos em tais eventos, comparado ao semestre anterior ao lançamento do site.

\subsection{Estudo de Caso 2: Melhoria na Divulgação de Artigos}
A seção de divulgação de artigos permitiu que os professores compartilhassem suas pesquisas com maior facilidade. Um professor relatou que, após a divulgação de um artigo no site, ele recebeu mais convites para palestras e colaborações, destacando a importância da visibilidade acadêmica proporcionada pela plataforma.

\chapter{Trabalhos Relacionados}
Diversas iniciativas similares foram identificadas no campo de estudo deste trabalho, as quais serão apresentadas a seguir. Estas poderão ser utilizadas como referências importantes para aprimoramento e desenvolvimento da nossa aplicação.\

\begin{enumerate}
\item A plataforma \emph{Graduação Digital}, disponível para web e dispositivos móveis, funciona desde 2019. Criada pela Universidade Federal de São Paulo, o site tem o objetivo de centralizar todas as informações relevantes sobre os cursos de graduação oferecidos. A plataforma fornece detalhes sobre os programas de estudo, cronograma de aulas, requisitos de graduação e suporte acadêmico. Uma funcionalidade importante é a integração com o sistema de matrícula, permitindo aos alunos gerenciar suas inscrições de maneira eficiente.

\item Em 2020, foi lançado o portal \emph{Docentes Conectados}, desenvolvido por Silva. A aplicação oferece uma solução inovadora para a comunicação e colaboração entre os membros do corpo docente. Além de perfis detalhados dos professores, com suas áreas de pesquisa e publicações, o portal inclui ferramentas para agendamento de reuniões, compartilhamento de recursos educativos e organização de grupos de estudo e pesquisa. Infelizmente, devido a cortes de financiamento, a plataforma enfrenta dificuldades de manutenção e atualizações regulares.

\item No ano de 2021, a Universidade Estadual do Ceará (UECE) lançou o projeto \emph{Extensão em Ação}, que visa promover os projetos de extensão desenvolvidos por estudantes e professores. O site permite a divulgação de projetos em andamento, resultados obtidos e oportunidades de participação para a comunidade acadêmica e externa. A plataforma também facilita a busca por parcerias e financiamento para novos projetos.

\item Em 2022, o então reitor da Universidade de Brasília (UnB), Dr. Joaquim Lima, apresentou à comunidade acadêmica a plataforma \emph{Artigos Acadêmicos UnB}. A ideia era criar um repositório digital que concentrasse todos os artigos publicados pelos professores da instituição, oferecendo uma interface amigável para busca e acesso a esses trabalhos. A plataforma inclui filtros por área de pesquisa, ano de publicação e autor, além de permitir o download dos artigos em formato PDF. O projeto atualmente continua em fase de expansão, incorporando novas funcionalidades.

\end{enumerate}
Nossa proposta de aplicação, por sua vez, se concentrará em integrar todas as informações mencionadas nas plataformas acima, além de desenvolver uma solução tecnológica \textit{gameficada}, a fim de tornar a experiência do gerenciamento acadêmico mais interativa e envolvente.

% Experiência do Usuário
\chapter{Experiência do Usuário}
 
\section{Iniciativa}
Para o site do curso de Sistemas para Internet, priorizar a UX significa garantir que tanto alunos quanto professores possam acessar informações de maneira eficiente, intuitiva e agradável. Este capítulo aborda os elementos de design e funcionalidade implementados para proporcionar uma experiência de usuário de alta qualidade.
 
\section{Design da Interface}
\subsection{Estética e Navegação}
A interface do site foi projetada com uma estética moderna e limpa, utilizando uma paleta de cores que não apenas favorece a legibilidade e o conforto visual, como também reafirma a identidade visual da nossa Universidade (Universidade Católica de Pernambuco) trazendo cores, fontes e imagens relacionadas ao curso de Graduação. A navegação é facilitada através de um menu principal fixo que permite acesso rápido às principais seções do site, como "Sobre o Curso - Sin", "Corpo Docente - Gestão e pessoas", "Pós-graduação", "Divulgação de Artigos" e "Pesquisa e extensão".

\subsection{Responsividade}
O site do curso de Sistemas para Internet foi desenvolvido com técnicas de design responsivo, garantindo que o layout e as funcionalidades se adaptem perfeitamente a diferentes tamanhos de tela, desde desktops até smartphones.
 
\subsection{Interatividade}
Elementos interativos, como menus dropdown, botões de ação e formulários dinâmicos, foram implementados para melhorar a interatividade do site. Além disso, a funcionalidade de login e registro é facilitada com formulários claros e feedback imediato sobre o status das ações do usuário.

\section{Impacto na Experiência Acadêmica}
\subsection{Engajamento dos Alunos}
O site facilita um maior engajamento dos alunos com as atividades do curso. A seção de "Extensões do Curso" promove uma maior participação em workshops, palestras e eventos, enquanto a disponibilidade de artigos científicos incentivou os alunos a se envolverem mais com a pesquisa acadêmica.
 
\subsection{Comunicação Eficiente}
A funcionalidade de avisos e comunicados irá melhorar significativamente a comunicação entre a coordenação do curso, os professores e os alunos. Informações críticas, como mudanças no cronograma e prazos de inscrição, são agora facilmente acessíveis, reduzindo o risco de desinformação e aumentando a eficiência administrativa.

% Conclusão
\chapter{Conclusão} 

\section{Considerações Finais} 
O desenvolvimento do site do curso de graduação em Sistemas para Internet buscou atender a uma necessidade de centralização e organização das informações pertinentes ao curso. Focamos em trazer dinamismo e interatividade à plataforma, proporcionando uma experiência de usuário rica e responsiva. 

Através deste projeto, foi possível observar a importância de uma comunicação eficiente e de fácil acesso entre alunos, professores e a administração do curso. A possibilidade de divulgar artigos científicos, avisos importantes e informações sobre eventos e atividades extracurriculares fortalece o vínculo entre os membros da comunidade acadêmica, promovendo um ambiente mais colaborativo e engajado. 

Em suma, o site do curso de Sistemas para Internet não apenas centraliza informações, mas também serve como uma plataforma de interação e troca de conhecimento, contribuindo significativamente para a qualidade do ensino e a satisfação dos alunos e professores.

\section{Trabalhos futuros no Site do curso de Sistemas para Internet} 
Apesar das funcionalidades implementadas, há diversas melhorias e expansões que podem ser realizadas no futuro para tornar a plataforma ainda mais robusta e abrangente. 

A participação ativa no site pode ser incentivada através de novas funcionalidades que promovam a interação entre os usuários. Por exemplo, a implementação de fóruns de discussão, onde alunos e professores possam debater sobre temas relevantes do curso, trocar experiências e resolver dúvidas, pode enriquecer a experiência acadêmica.  

Além disso, a criação de um sistema de feedback para as aulas e professores pode proporcionar uma valiosa fonte de informações para a melhoria contínua do ensino. 

Outra iniciativa importante seria a introdução de projetos colaborativos online, onde alunos de diferentes turmas possam trabalhar juntos em projetos de desenvolvimento, pesquisa ou inovação. Isso não só aumenta o engajamento, mas também prepara os alunos para o trabalho em equipe, uma habilidade essencial no mercado de trabalho atual. 

\subsection{Recomendações para nossos usuários} 
Para garantir que os usuários aproveitem ao máximo as funcionalidades do site, algumas recomendações são essenciais: 

\begin{itemize} 
    \item \textbf{Explorar todas as seções:} Alunos e professores devem familiarizar-se com todas as seções do site, especialmente aquelas dedicadas a avisos, artigos científicos e informações sobre o corpo docente. Isso garante que todos estejam atualizados sobre as últimas novidades e recursos disponíveis. 

    \item \textbf{Participar de atividades extracurriculares:} A seção de extensões do curso oferece uma variedade de atividades extracurriculares que são fundamentais para o desenvolvimento profissional e pessoal dos alunos. Participar de workshops, palestras e projetos de pesquisa pode proporcionar uma experiência educacional mais rica e diversificada. 

    \item \textbf{Utilizar as funcionalidades de interação:} Ferramentas de interação, como fóruns de discussão (caso implementados futuramente), podem ser muito úteis para a resolução de dúvidas e para a troca de conhecimentos. Incentiva-se a participação ativa nessas plataformas. 

    \item \textbf{Fornecer feedback:} O feedback dos usuários é crucial para a melhoria contínua da plataforma. Alunos e professores devem ser encorajados a reportar problemas, sugerir melhorias e compartilhar suas experiências com a equipe de desenvolvimento.

\end{itemize}
Em conclusão, o site do curso de Sistemas para Internet já oferece uma base sólida e funcional, mas é essencial continuar evoluindo e adaptando-se às necessidades dos usuários. Com a participação ativa da comunidade acadêmica e a implementação de novas funcionalidades, a plataforma pode tornar-se um recurso ainda mais valioso e indispensável para todos os envolvidos.

\end{document}
